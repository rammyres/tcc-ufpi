\chapter{Revisão Bibliográfica}
%---------------------
% Edite a partir daqui 
%---------------------
Para editar a revisão bibliográfica edite o arquivo revisão.tex na pasta textuais. 
\par
Para criar a lista de referência recomendo criar um arquivo bibtex )referencias.bib) no Mendeley, Jabref, bibtex.online, ou outro gerenciador de referências, ou ainda criar as referências a partir dos geradores de sites como Google Scholar, Periodicos Capes, etc. 
\par
Para citar no estilo (AUTOR, 2000) use \verb+\cite{id da referência}+
\par
Por exemplo: \cite{Abuidris2019}
\par
Para citar no estilo Autor (2000) use \verb+\citeonlin{id da referência}+
\par
Por exemplo: \citeonline{Ufpi2020}
\par
As ids de referências são as existentes no começo de cada entrada do arquivo bibtex. As citações criam as referências automaticamente no fim do texto, somente aquelas existentes no arquivo referencia.bib e efetivamente usadas aparecerão na seção correspondente.

Para criar apêndices e anexos, edite a parte correspondente no arquivo TCC1.tex, removendo os comentários \verb+%+ nos comandos \verb+\partapendices+ e \verb+\partanexos+ da seção postextuais, e incluindo os arquivos pertinentes na pasta postextuais.
