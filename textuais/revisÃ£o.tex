\chapter{Revisão Bibliográfica}
%---------------------
% Edite a partir daqui 
%---------------------
Para editar a revisão bibliográfica edite o arquivo revisão.tex na pasta textuais. 
\par
Para criar a lista de referência recomendo criar um arquivo bibtex no Mendeley, Jabref, bibtex.online, ou outro gerenciador de referências, ou ainda criar as referências a partir dos geradores de sites como Google Scholar, Periodicos Capes, etc. 
\par
Para chamar uma citação você deve usar o \verb+\cite{id da referência}. ou \citeonlin{id da referência}+
\par
Para citar no estilo (Autor, 2000) use \verb+\cite{id da referência}+
\par
Por exemplo: \cite{Abuidris2019}
\par
Para citar no estilo Autor (2000) use \verb+\citeonlin{id da referência}+
\par
Por exemplo: \citeonline{Ufpi2020}
\par
As citações criam as referências automaticamente no fim do texto, somente referências existentes no arquivo referencia.bib aparecerão na referências.

Para criar apêndices e anexos, edite a parte correspondente no arquivo TCC1.tex, removendo os comentários \verb+%+ nos comandos \verb+\partapendices+ e \verb+\partanexos+ da seção postextuais, e incluindo os arquivos pertinentes na pasta postextuais.

Por exemplo, para um anexo inclua: \verb+\include{postextuais/anexo.tex}+ par incluir um anexo. 