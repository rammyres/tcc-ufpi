\chapter{Problematização}
%---------------------
% Edite a partir daqui 
%---------------------
Para editar a problematização, edite o arquivo problematização.tex na pasta textuais. 

\section{Seção}
Para criar uma seção use o comando:
	\begin{verbatim}
		\section{Nome da seção}
	\end{verbatim}

\section{Figuras}

Para incluir figuras utilize:

\begin{verbatim}
	\begin{figure}[!htb]
		\centering 
		\includegraphics[width=0.5\textwidth]{imagens/ufpi}
		\caption{Brasao da UFPI}
		\label{fig:BrasaoUFPI}
	\end{figure}
\end{verbatim}


\begin{figure}[!htb]
	\centering 
	\includegraphics[width=0.5\textwidth]{imagens/ufpi}
	\caption{Brasão da UFPI}
	\label{fig:BrasaoUFPI}
\end{figure}
\clearpage

O nome do arquivo imagem não deve conter a extensão do arquivo, o \LaTeX \ processará automaticamente se o arquivo for suportado. 

O comando \verb+\textwidth+ retorna a largura da página e ao utilizar ele a imagem ocupara toda extensão dela. Para reduzir a largura utilize um multiplicador fracionário ou um valor absoluto em centímetros (cm) ou milímetros (mm), como o exemplo a seguir:

\begin{figure}[!htb]
	\centering 
	\includegraphics[width=8cm]{imagens/latex}
	\caption{Logo \LaTeX}
	\label{fig:Latex}
\end{figure}

\begin{figure}[!htb]
	\centering 
	\includegraphics[width=100mm]{imagens/latex}
	\caption{Logo \LaTeX}
	\label{fig:Latex2}
\end{figure}

O comando \verb+\caption{}+ inclui a legenda a foto e o \verb|\label{fig:nome_da_figura| inclui uma  referência para a imagem na tabela de ilustrações. 

Atente a posição das figuras, se o resultado da página ficar desorganizado, utilize \verb|\clearpage| para evitar que a imagem se distancie demais do texto a que se refere. 

\subsection{Subseção}
Subseções são marcadas com \verb|\subsection{Nome da subseção}|. 
