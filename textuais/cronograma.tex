\chapter{Cronograma}
%---------------------
% Edite a partir daqui 
%---------------------
Para editar o cronograma você deve editar o arquivo cronograma.tex na pasta textuais. No entanto recomendo usar um editor de tabelas para latex. 
\begin{table}[h]
	\centering
	\small
	\begin{tabular}{|l|l|l|l|l|l|l|l|} 
		\hline
		& Ago 2020 & Set 2020 & Out 2020 & Nov 2020 & Dez 2020 & Jan 2021 & Fev 2020  \\ 
		\hline
		\begin{tabular}[c]{@{}l@{}}Desenhar e \\prototipar o \\sistema\end{tabular} & \cellcolor{blue!25}          &          &          &          &          &          &           \\ 
		\hline
		\begin{tabular}[c]{@{}l@{}}Implementar o \\sistema\end{tabular}             &          & \cellcolor{blue!25}         &          &          &          &          &           \\ 
		\hline
		\begin{tabular}[c]{@{}l@{}}Realizar eleições \\simuladas\end{tabular}     &          &          &  \cellcolor{blue!25}        & \cellcolor{blue!25}         &          &          &           \\ 
		\hline
		\begin{tabular}[c]{@{}l@{}}Coletar dados\end{tabular}                    &          &          & \cellcolor{blue!25}         & \cellcolor{blue!25}         &          &          &           \\ 
		\hline
		\begin{tabular}[c]{@{}l@{}}Analisar dados\end{tabular}                   &          &          &          &          & \cellcolor{blue!25}         &          &           \\ 
		\hline
		\begin{tabular}[c]{@{}l@{}}Escrever \\a monografia\end{tabular}             &          &          &          &          &  \cellcolor{blue!25}        &  \cellcolor{blue!25}        &           \\ 
		\hline
		\begin{tabular}[c]{@{}l@{}}Defender \\a monografia\end{tabular}             &          &          &          &          &          &          & \cellcolor{blue!25}           \\
		\hline
	\end{tabular}
\caption{Cronograma}
\label{table:tabela1}
\end{table}
