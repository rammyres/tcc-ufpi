%% abtex2-modelo-projeto-pesquisa.tex, v-1.9.7 laurocesar
%% Copyright 2012-2018 by abnTeX2 group at http://www.abntex.net.br/ 
%%
%% This work may be distributed and/or modified under the
%% conditions of the LaTeX Project Public License, either version 1.3
%% of this license or (at your option) any later version.
%% The latest version of this license is in
%%   http://www.latex-project.org/lppl.txt
%% and version 1.3 or later is part of all distributions of LaTeX
%% version 2005/12/01 or later.
%%
%% This work has the LPPL maintenance status `maintained'.
%% 
%% The Current Maintainer of this work is the abnTeX2 team, led
%% by Lauro César Araujo. Further information are available on 
%% http://www.abntex.net.br/
%%
%% This work consists of the files abntex2-modelo-projeto-pesquisa.tex
%% and abntex2-modelo-references.bib
%%

% ------------------------------------------------------------------------
% ------------------------------------------------------------------------
% abnTeX2: Modelo de Projeto de pesquisa em conformidade com 
% ABNT NBR 15287:2011 Informação e documentação - Projeto de pesquisa -
% Apresentação 
% ------------------------------------------------------------------------ 
% ------------------------------------------------------------------------
\documentclass[
	% -- opções da classe memoir --
	12pt,				% tamanho da fonte
	openright,			% capítulos começam em pág ímpar (insere página vazia caso preciso)
	twoside,			% para impressão em recto e verso. Oposto a oneside
	a4paper,			% tamanho do papel. 
	% -- opções da classe abntex2 --
	%chapter=TITLE,		% títulos de capítulos convertidos em letras maiúsculas
	%section=TITLE,		% títulos de seções convertidos em letras maiúsculas
	%subsection=TITLE,	% títulos de subseções convertidos em letras maiúsculas
	%subsubsection=TITLE,% títulos de subsubseções convertidos em letras maiúsculas
	% -- opções do pacote babel --
	english,			% idioma adicional para hifenização
	french,				% idioma adicional para hifenização
	spanish,			% idioma adicional para hifenização
	brazil,				% o último idioma é o principal do documento
	]{abntex2}

% ---
% PACOTES
% ---

% ---
% Pacotes fundamentais 
% ---
\usepackage{helvet}			% Usa a fonte Helvetica
\usepackage[T1]{fontenc}		% Selecao de codigos de fonte.
\usepackage[utf8]{inputenc}		% Codificacao do documento (conversão automática dos acentos)
\usepackage{indentfirst}		% Indenta o primeiro parágrafo de cada seção.
\usepackage{color}				% Controle das cores
\usepackage{graphicx}			% Inclusão de gráficos
%\usepackage{microtype} 			% para melhorias de justificação
\usepackage[nonumberlist=true, 
					toc,
					style=index,
					translate=babel,
					automake]{glossaries}
%\makeglossaries
% ---

% ---
% Pacotes adicionais, usados apenas no âmbito do Modelo Canônico do abnteX2
% ---
% ---
% Pacotes de citações
% ---
\usepackage[brazilian,hyperpageref]{backref}	 % Paginas com as citações na bibl
\usepackage[alf, abnt-emphasize=bf, bibjustif]{abntex2cite}	% Citações padrão ABNT
\usepackage[table,xcdraw]{xcolor} % Permite a colorizaão da tabela do cronograma
\usepackage{lscape} % Permite a impressão de páginas em modo paisagem 
\usepackage{hyperref}
\usepackage{diagbox}
\usepackage{verbatim}

% --- 
% CONFIGURAÇÕES DE PACOTES
% --- 

% ---
% Configurações do pacote backref
% Usado sem a opção hyperpageref de backref
\renewcommand{\backrefpagesname}{Citado na(s) página(s):~}
% Texto padrão antes do número das páginas
\renewcommand{\backref}{}
% Define os textos da citação
\renewcommand*{\backrefalt}[4]{
	\ifcase #1 %
		Nenhuma citação no texto.%
	\or
		Citado na página #2.%
	\else
		Citado #1 vezes nas páginas #2.%
	\fi}%
% ---

% ---
% Informações de dados para CAPA e FOLHA DE ROSTO
% ---

\titulo{Edite o arquivo dados.tex para colocar os dados da capa, folha de rosto e aprovação}
\autor{Nome do autor}
\local{Cidade do Autor - PI}
\data{mês e ano}
\instituicao{%
  UNIVERSIDADE FEDERAL DO PIAUÍ -- UFPI
  \par
  CENTRO DE EDUCAÇÃO ABERTA E A DISTÂNCIA – CEAD/UFPI
  \par
  CURSO DE BACHARELADO EM SISTEMAS DE INFORMAÇÃO}
\tipotrabalho{Trabalho de Conclusão de Curso}
% O preambulo deve conter o tipo do trabalho, o objetivo, 
% o nome da instituição e a área de concentração 
\preambulo{Monografia submetida ao Curso de Bacharelado de Sistemas de Informação como requisito parcial para obtenção de grau de Bacharel em Sistemas de Informação.}
\orientador{Nome do Orientador}
% Comente a linha abaixo com '%' se não houver coorientador
\coorientador{Nome do Coorientador}
% ---
% Para multiplos coorientadores use a seguinte alternativa:
% \coorientador{Nome do Coorientador 1\par Coorientador: Nome do Coorientador 2}
% ---

\loadglsentries{postextuais/glossarios} %Carrega as entradas do glossário, se existirem
\makeglossaries

% ---
% Configurações de aparência do PDF final
% ---


% ----e um
% INICIO DAS CUSTOMIZACOES PARA A UNIVERSIDADE FEDERAL DO PIAUÍ
% ---

\newcommand{\capaufpi}{%
  \begin{capa}%
	\center
	\textbf{\normalsize\imprimirinstituicao}
	   
	    \vfill
	    \begin{vplace}[0.2]
	    \textbf
		    {\large{MONOGRAFIA}}
	    \vspace*{1.0cm}
		\textbf{\center\LARGE\imprimirtitulo}
		     
	    \vspace{2cm}
	    \textbf{\large\imprimirautor}
		\end{vplace}
		
		\vfill
		    
		\textbf{\normalsize\imprimirlocal}
		
		\textbf{\normalsize\imprimirdata}
	    
	    \vspace*{1cm}

  \end{capa}
}


% folha de rosto 

\makeatletter

\newcommand{\frostoUFPI}{
\begin{center}
	%\large\imprimirinstituicao
    
%    \vspace*{1cm}
    
{\large\imprimirautor}

\vspace*{\fill}\vspace*{\fill}

\begin{center}
\bfseries\Large\imprimirtitulo
\end{center}

\vspace*{\fill}

\abntex@ifnotempty{\imprimirpreambulo}{%
  \hspace{.45\textwidth}
  \begin{minipage}{.5\textwidth}
  \SingleSpacing
  \imprimirpreambulo
  \par
  \vspace{1cm}
  \imprimirorientadorRotulo~\imprimirorientador\par
  \imprimircoorientadorRotulo~\imprimircoorientador
  \end{minipage}%
  \vspace*{\fill}
}%



	\vspace*{\fill}


	{\large\imprimirlocal}

	\par

	{\large\imprimirdata}
	\vspace*{1cm}
\end{center}
}

\makeatother

% ---
% FIM DAS CUSTOMIZACOES PARA A UNIVERSIDADE FEDERAL DO PIAUÍ
% ---

% alterando o aspecto da cor azul
\definecolor{blue}{RGB}{41,5,195}

% informações do PDF
\makeatletter
\hypersetup{
     	%pagebackref=true,
		pdftitle={\@title}, 
		pdfauthor={\@author},
    	pdfsubject={\imprimirpreambulo},
	    pdfcreator={LaTeX with abnTeX2},
		pdfkeywords={abnt}{latex}{abntex}{abntex2}{projeto de pesquisa}, 
		colorlinks=true,       		% false: boxed links; true: colored links
    	linkcolor=blue,          	% color of internal links
    	citecolor=blue,        		% color of links to bibliography
    	filecolor=magenta,      		% color of file links
		urlcolor=blue,
		bookmarksdepth=4
}
\makeatother
% --- 

% --- 
% Espaçamentos entre linhas e parágrafos 
% --- 

% O tamanho do parágrafo é dado por:
\setlength{\parindent}{1.3cm}

% Controle do espaçamento entre um parágrafo e outro:
\setlength{\parskip}{0.2cm}  % tente também \onelineskip

% ---
% compila o indice
% ---
\makeindex
% ---
% ----
% Início do documento
% ----
\begin{document}

% Seleciona o idioma do documento (conforme pacotes do babel)
%\selectlanguage{english}
\selectlanguage{brazil}

% Retira espaço extra obsoleto entre as frases.
\frenchspacing 

% ----------------------------------------------------------
% ELEMENTOS PRÉ-TEXTUAIS
% ----------------------------------------------------------
% \pretextual

% ---
% Capa
% ---
%\imprimircapa
\capaufpi
% ---

% ---
% Folha de rosto
% ---
%\imprimirfolhaderosto
\frostoUFPI
% ---
% ================================================================================
% Modelo da folha de aprovação, padrão UFPI, conforme modelo 
% disponível em https://drive.google.com/file/d/0B0BKqBVXhtedLWQzYVBhbUNRT1U/edit
% ================================================================================

\begin{folhadeaprovacao}
\begin{center}
	{\ABNTEXchapterfont\large\imprimirautor}
	\vspace*{\fill}\vspace*{\fill}
	\begin{center}
		\ABNTEXchapterfont\bfseries\Large\imprimirtitulo
	\end{center}
	\vspace*{\fill}
	\hspace{.45\textwidth}
	\begin{minipage}{.5\textwidth}
		\imprimirpreambulo
	\end{minipage}%
	\vspace*{\fill}
\end{center}

Trabalho \rule{4cm}{1pt}. \imprimirlocal, \rule{1cm}{1pt} de \rule{3cm}{1pt} de 2020:

% ===
% edite os nomes e cargos dos professores da banca
% ===
\assinatura{\textbf{\imprimirorientador} \\ Orientador}
\assinatura{\textbf{Nome do coorientador} \\ Coorientador}
\assinatura{\textbf{Professor} \\ Convidado 1}
\assinatura{\textbf{Professor} \\ Convidado 2}
\assinatura{\textbf{Professor} \\ Convidado 3}
\begin{center}
	\vspace*{0.5cm}
	{\large\imprimirlocal}
	\par
	{\large\imprimirdata}
	\vspace*{1cm}
\end{center}
\end{folhadeaprovacao}
% ---
% NOTA DA ABNT NBR 15287:2011, p. 4:
%  ``Se exigido pela entidade, apresentar os dados curriculares do autor em
%     folha ou página distinta após a folha de rosto.''
% ---

% ---
% inserir lista de ilustrações
% ---
%\pdfbookmark[0]{\listfigurename}{lof}
%\listoffigures*
\cleardoublepage
% ---

% ---
% inserir lista de tabelas
% ---
%\pdfbookmark[0]{\listtablename}{lot}
\listoftables*
\cleardoublepage
% ---

% ---
% inserir lista de abreviaturas e siglas
% Edite o arquivo siglas.tex na pasta pretextuais
% ---
\begin{siglas}
  \item[UFPI] Universidadde Federal do Piauí
  \item[CEAD] Centro de Educação Aberta e a Distância
\end{siglas}
% ---

% ---
% inserir lista de símbolos
% ---
%\begin{simbolos}
%  \item[$ \Gamma $] Letra grega Gama
%  \item[$ \Lambda $] Lambda
%  \item[$ \zeta $] Letra grega minúscula zeta
%  \item[$ \in $] Pertence
%\end{simbolos}
% ---

% ---
% inserir o sumario
% ---
\pdfbookmark[0]{\contentsname}{toc}
\tableofcontents*
\cleardoublepage
% ---

% ----------------------------------------------------------
% ELEMENTOS TEXTUAIS
% ----------------------------------------------------------
\textual
% ----------------------------------------------------------
% Os includes a seguir incluirão os arquivos contidos na pasta 
% textuais. 
% ----------------------------------------------------------
\begin{resumo}
	%Resumo
	\vspace{\onelineskip}
	\noindent
	Para editar o resumo edite o arquivo resumo.tex na pasta textuais. 
	\par\textbf{Palavras-chave}: resumo, latex
\end{resumo}



\begin{resumo}[Abstract]
\begin{otherlanguage*}{english}  
	\vspace{\onelineskip}
	\noindent
		%---------------------
		% Edite a partir daqui 
		%---------------------
		To edit the abstract, edit the file abstract.tex in the textuais folder. 
		\par\textbf{Keywords}: latex, UFPI.
\end{otherlanguage*}
\end{resumo}
\chapter{Introdução}
%---------------------
% Edite a partir daqui 
%---------------------
Para editar a introdução, edite o arquivo introdução.tex na pasta textuais. 
\chapter{Justificativa}
%---------------------
% Edite a partir daqui 
%---------------------
Para editar a justificativa, edite o arquivo justificativa.tex na pasta textuais. 
\chapter{Problematização}
%---------------------
% Edite a partir daqui 
%---------------------
Para editar a problematização, edite o arquivo problematização.tex na pasta textuais. 
\chapter{Hipotese}
%---------------------
% Edite a partir daqui 
%---------------------
Para editar a hipotese, edite o arquivo hipotese.tex na pasta textuais. 
\chapter{Objetivos}
%---------------------
% Edite a partir daqui 
%---------------------

Para editar os objetivos, edite o arquivo objetivos.tex. 


\begin{itemize}
	\item Este é um modelo de lista não-numerada.
\end{itemize}



\include{textuais/revisão}
\chapter{Metodologia}
%---------------------
% Edite a partir daqui 
%---------------------
Para alterar a metodologia, edite o arquivo metodologia.tex na pasta textuais. 

\chapter{Cronograma}
%---------------------
% Edite a partir daqui 
%---------------------
Para editar o cronograma você deve editar o arquivo cronograma.tex na pasta textuais. No entanto recomendo usar um editor de tabelas para latex. 
\begin{table}[h]
	\centering
	\small
	\begin{tabular}{|l|l|l|l|l|l|l|l|} 
		\hline
		& Ago 2020 & Set 2020 & Out 2020 & Nov 2020 & Dez 2020 & Jan 2021 & Fev 2020  \\ 
		\hline
		\begin{tabular}[c]{@{}l@{}}Desenhar e \\prototipar o \\sistema\end{tabular} & \cellcolor{blue!25}          &          &          &          &          &          &           \\ 
		\hline
		\begin{tabular}[c]{@{}l@{}}Implementar o \\sistema\end{tabular}             &          & \cellcolor{blue!25}         &          &          &          &          &           \\ 
		\hline
		\begin{tabular}[c]{@{}l@{}}Realizar eleições \\simuladas\end{tabular}     &          &          &  \cellcolor{blue!25}        & \cellcolor{blue!25}         &          &          &           \\ 
		\hline
		\begin{tabular}[c]{@{}l@{}}Coletar dados\end{tabular}                    &          &          & \cellcolor{blue!25}         & \cellcolor{blue!25}         &          &          &           \\ 
		\hline
		\begin{tabular}[c]{@{}l@{}}Analisar dados\end{tabular}                   &          &          &          &          & \cellcolor{blue!25}         &          &           \\ 
		\hline
		\begin{tabular}[c]{@{}l@{}}Escrever \\a monografia\end{tabular}             &          &          &          &          &  \cellcolor{blue!25}        &  \cellcolor{blue!25}        &           \\ 
		\hline
		\begin{tabular}[c]{@{}l@{}}Defender \\a monografia\end{tabular}             &          &          &          &          &          &          & \cellcolor{blue!25}           \\
		\hline
	\end{tabular}
\end{table}


% ----------------------------------------------------------
% Capitulo com exemplos de comandos inseridos de arquivo externo 
% ----------------------------------------------------------

%\include{abntex2-modelo-inclCitaude-comandos}

% ---
% Finaliza a parte no bookmark do PDF
% para que se inicie o bookmark na raiz
% e adiciona espaço de parte no Sumário
% ---
\phantompart

% ---
% Conclusão
% ---

% ----------------------------------------------------------
% ELEMENTOS PÓS-TEXTUAIS
% ----------------------------------------------------------
\postextual

% ----------------------------------------------------------
% Referências bibliográficas
% ----------------------------------------------------------
\bibliography{postextuais/referencias}

% ----------------------------------------------------------
% Glossário
% ----------------------------------------------------------
%
% Consulte o manual da classe abntex2 para orientações sobre o glossário.
%

%\printglossaries

\glossary{postextuais/glossarios}
\printglossaries

% ----------------------------------------------------------
% Apêndices
% ----------------------------------------------------------

% ---
% Inicia os apêndices
% ---
\begin{apendicesenv}

% Imprime uma página indicando o início dos apêndices
% ↓ Remova o '%' para imprimir o inicio da parte dos apêndices
%\partapendices

% O que for incluido por aqui até o end serão os aprendices

\end{apendicesenv}
% ---


% ----------------------------------------------------------
% Anexos
% ----------------------------------------------------------

% ---
% Inicia os anexos
% ---
\begin{anexosenv}

% Imprime uma página indicando o início dos anexos
% ↓ Remova o '%' para imprimir o inicio da parte dos anexos
  %\partanexos
 
%O que for incluido por aqui até o end será considerado anexo

\end{anexosenv}

%---------------------------------------------------------------------
% INDICE REMISSIVO
%---------------------------------------------------------------------

\phantompart

\printindex


\end{document}
